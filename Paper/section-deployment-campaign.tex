\section{Deployment}

The deployment of the first prototype started in late spring of 2014.
A few pilot sites with close association to project members were
chosen and contacted. The goal of this campaign was to collect and
upload to the central data service the storage information from a few
sites. This allowed to independently verify the provided client
software tools and that the documentation is complete and well
structured. It also provided some first data that could be used to
design and test the visualization tools. Administrators from several
sites contributed with feedback and improvements in this first test.
The pilot deployment led to enhancements in the central data service
and to updates of the documentation.

In February of 2015 CMS computing management decided to proceed with
the role out of the space monitoring. CMS site support setup a metric
in the CMS dashboard that monitors the space information uploaded by
sites. When this was ready, sites were contacted in small batches and
ask to setup regular uploading of storage information to the central
data service. About ten site administrators were contacted via email
each week. This allowed to sort out any remaining hidden issues, fine
tune instructions, and provide prompt response to questions and in
case of problems. The last sites were contacted beginning of May. The
issues encountered during this first stage of this deployment can be
categorized into three groups: 1) Questions from sites about why they
need to provide storage usage information and at what level of detail;
2) Authentication problems uploading the information to the central
data service; 3) The long time it takes to take a dump for some
storage systems. Sites were informed about how CMS plans to use the
storage information. Site were reminded about the three main reasons
for the space monitoring project: i) consistency checks between site
storage and central PhEDEx catalogue, ii) monitoring the inventory of
areas that are not managed by PhEDEx, and iii) to apply user and group
quotas fairly at the experiment lavel, i.e.\ across sites. For level
of detail on the storage information guidance was provided and the
documentation clarified. The by far largest issues were related to
the authentication to upload the information. Instructions and also a
script to troubleshoot the authentication problems has been provided.
In the process we found the perl-based upload client to not function
correctly with the one of the perl-SSL libraries and a problem with
the curl command in the current Scientific Linux distributions. On
some large storage systems a dump can take several days to complete
and this needs to be accounted for. As of middle May 2015, about half
of the sites are uploading space information regularly. Several sites
are in the process of setting it up. There are no outstanding
technical issues preventing sites from providing space information.
The plan is to follow the first stage with a ticketing campaign for
any remaining sites. We anticipate the roll out to be completed in
June.

With plans to enhance the schema of the central data service, sites
have been ask to keep the storage dumps they currently take. Given
the upload issues, we are also looking into at other secure upload
solutions.
