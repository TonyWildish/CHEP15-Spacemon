\section {Site information providers}


CMS sites use various storage technologies for maintaining CMS data: Castor, dCache,DPM, EOS, Hadoop, LStore, Lustre, StoRM.There is no unique solution for retrieving the local storage information for all sites. On the posix-compliant file systems, such as EOS, Lustre, or hadoop systems mounted with FUSE, the find command can do the work.For dCache, direct database query on the namespace database is often more efficient and causes less load on the system than traversing the whole namespace via NFS client tools. The impact on the production storage namespace can be completely avoided if the storage dumps are produced on the hot standby server, onto which the namespace database is dynamically replicated for the backup purpose.Sites sharing the same storage instance between multiple virtual organizations may optimize by dumping only a sub-branch of the namespace under CMS-owned directory.Another possible way to optimize is to do aggregation on the database level instead of producing the whole storage dump. SpaceMon client provides an API to read in the aggregated record in a pre-defined format.CMS site admins exchange tools and solutions for creating storage dumps via an open source repository on github [2]. 

\begin{figure}[h]
\center
\includegraphics[width=1.0\linewidth]
{pictures/sites_contributions.pdf}
\caption{Commits contributed by CMS site admins to SiteInfoProvider tools repository}
\label{fig:github_stats}
\end{figure}

